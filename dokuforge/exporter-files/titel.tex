\hypersetup {
  pdftitle    = {CdE AKADEMIE JAHR}
}

\title {CdE AKADEMIE JAHR}

\date{}

\publishers%
  {\includegraphics[width=0.8\linewidth]{images/fig_logo}\\%das muss jeweils an das konkrete Logo angepasst werden, es soll schließlich optisch mittig aussehen
   \bigskip

    \Large \CdEeV}

\uppertitleback%
  {Mit freundlicher Unterstützung durch die SPONSOREN (mit Logos; sofern vorhanden)}

\lowertitleback%
  {\smaller%
   Die in dieser Dokumentation enthaltenen Texte wurden von den Kursleitern
   und Teilnehmern der CDE AKADEMIE JAHR erstellt.
   \medskip

   \begin{minipage}[b]{0.8\textwidth}
   \emph{Akademiefuzzi:} NAME
   \smallskip

   \emph{Redaktion:}
    Red Akteur, Korrektur Leserin, Bild Bearbeiter, Red Akteurin, Korrektur Leser, Bild Bearbeiterin,
    Red Akteur, Korrektur Leserin, Bild Bearbeiter, Red Akteurin, Korrektur Leser, Bild Bearbeiterin,
    Red Akteur, Korrektur Leserin, Bild Bearbeiter, Red Akteurin, Korrektur Leser, Bild Bearbeiterin
   \smallskip

   \emph{Endredaktion:} NAMEN
   \end{minipage}
   \hfill
   \raisebox{0.5\baselineskip}{\includegraphics[height=23mm]{Fuzzi-Hut-Logo}}\hspace*{-4.1mm}
   % raisebox: vertikal positioniert für 4 Redaktionszeilen
   % hspace: optisch den unteren Punkt an den rechten Seitenrand
   \smallskip

   Dieses Dokument wurde mit Hilfe von \LaTeX\ gesetzt. Als
   Hauptschriften wurden die Linotype Palatino von Hermann Zapf und die
   Mathpazo von Diego Puga verwendet.
   \smallskip

   \emph{Druck und Bindung:} K+K-Copy-Druck-Service, Heidelberg
   \smallskip

   Copyright \textcopyright\ JAHR \CdEeV, Bonn. % Jahr ist Jahr der Akademie
   Alle Rechte vorbehalten.
}

\maketitle

\endinput
